\documentclass{beamer}
\usepackage{xcolor}
\usepackage{amsmath}
\usepackage{hyperref}
\hypersetup
{
colorlinks=true,
linkcolor=blue,
filecolor=magenta,      
urlcolor=cyan,
pdftitle={\LaTeX{} Basics \& Advanced},
pdfpagemode=FullScreen,
}
\usepackage[utf8]{inputenc}

\usetheme{Madrid}
\usecolortheme{default}


\title[\LaTeX{} Basics \& Advanced] 
{\LaTeX{} Basics \& Advanced}


\author[Ashwin Abraham] 
{Ashwin Abraham}

\institute[IIT-B] 
{
    IIT Bombay
}

\date[2022]
{August 2022}

\logo{\includegraphics[height=2cm]{iitb-logo.png}}


\begin{document}
    \frame{\titlepage}

    \section{Introduction}
    {
        \begin{frame}
            \frametitle{Introduction}
                \texttt{\$ whoami\pause \\ \$ ashwinabraham}


            \pause
            Who am I? 
            You sure you want to know? 
            The story of my life is not for the faint of heart. 
            If somebody said it was a happy little tale... 
            if somebody told you I was just your average ordinary guy, not a care in the world... somebody lied.\footnote{Adapted from \textit{Spider-Man (2002)}}
        \end{frame}

        \begin{frame}
            \frametitle{Table of Contents}
            \tableofcontents
    
            Ayy! I can redirect to stuff :D
        \end{frame}
        
        \begin{frame}
            \frametitle{Introduction to \LaTeX{}}

            What \textbf{\LaTeX{}} is\footnote{Adapted from the \href{https://en.wikipedia.org/wiki/LaTeX}{Wikipedia article on \LaTeX{}}}:

            \textbf{\LaTeX{}} is a software system for document preparation. 
            When writing, the writer uses \textbf{plain text} \pause as opposed to the \textit{formatted text} found in \texttt{WYSIWYG} word processors like \textit{Microsoft Word}, \textit{LibreOffice Writer} and \textit{Apple Pages}. 
            \pause
            The writer uses markup tagging conventions to define the general structure of a document \pause to \textcolor{red}{stylise text throughout a document} (such as \textbf{bold} and \textit{italics})\pause, and to add citations and cross-references. 
            \pause
            A TeX distribution such as TeX Live or MiKTeX is used to produce an output file (such as PDF or DVI) suitable for printing or digital distribution.

        \end{frame}
    }

    \section{Equations}
    {
        \begin{frame}
            \frametitle{Equations}
            We can also write equations in \LaTeX{}!!

            A labelled equation:
            \pause
            \begin{equation}
                e^{i\alpha} = \cos(\alpha) + i\sin(\alpha)
            \end{equation}
            \pause
            An unlabelled equation:
            \pause
            \begin{equation*}
                F = \frac{1}{4\pi\epsilon_{0}} \frac{q_{1}q_{2}}{r^{2}}
            \end{equation*}
        \end{frame}
    }

    \section{Itemization and Linking}
    {
        \begin{frame}
            \frametitle{Some Sorting Algorithms}
            There exist many sorting algorithms.
            \pause

            Some algorithms run in $O(n^{2})$, such as:
            \pause
            \begin{itemize}
                \item Bubble Sort
                \item Selection Sort
            \end{itemize}
            \pause
            Some run in $O(n\log(n))$, such as:
            \pause
            \begin{itemize}
                \item Merge Sort
                \item Quick Sort
            \end{itemize}
            \pause
            Some run in $O(n)$ (for some arrays), such as:
            \pause
            \begin{itemize}
                \item Counting Sort
            \end{itemize}
            \pause
            And some are \textcolor{pink}{joke algorithms}, such as:
            \begin{itemize}
                \item <8-> \textit{\href{https://en.wikipedia.org/wiki/Bogosort}{Bogosort}}
                \item <9-> \textit{\href{https://quantumcomputing.stackexchange.com/questions/1265/what-can-we-learn-from-quantum-bogosort}{Quantum Bogosort}}
                \item <10-> \textit{\href{https://en.wikipedia.org/wiki/Slowsort}{Slowsort}}
            \end{itemize}
            which are \textbf{way worse} than even $O(n^{2})$!
        \end{frame}
    }

    \section{Matrices and Indented Equations}
    {
        \begin{frame}
            \frametitle{Matrices and Indented Equations}
            We can also write matrices in \LaTeX{}!

            For example the (3x3) Identity Matrix can be written as:
            \begin{equation*}
                I_{3} = \begin{bmatrix}
                    1 & 0 & 0 \\
                    0 & 1 & 0 \\
                    0 & 0 & 1
                \end{bmatrix}
            \end{equation*}
            \pause
            I can also indent equations, like so:
            \begin{align*}
                (\textbf{a}\cdot\textbf{b})^{2} &= (\sum a_{i}b_{i})^{2} \\
                &\leq (\sum a_{i})^{2}(\sum b_{i})^{2}
            \end{align*}
        \end{frame}
    }
\end{document}